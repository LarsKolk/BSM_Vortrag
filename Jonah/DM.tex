\documentclass[11pt,a4paper,titlepage]{beamer}
\usepackage[utf8]{inputenc}
\usepackage{babel}
\usepackage{amsmath}
\usepackage{amsfonts}
\usepackage{graphicx}
\usepackage{amssymb}
\usepackage{sidecap}
\usepackage{hyperref}
\usepackage{siunitx}
\usepackage{booktabs}
\usepackage{animate}
%\usepackage{multimedia}
\usepackage{selinput}
\usepackage{media9}
\usepackage{tikz}
\usepackage{pgfplots}
\usepackage{subcaption}
\usepackage{appendixnumberbeamer}
\usepackage[
  backend=biber,   % use modern biber backend
  autolang=hyphen, % load hyphenation rules for if language of bibentry is not
  style=numeric,                 % german, has to be loaded with \setotherlanguages
  sorting=none                 % in the references.bib use langid={en} for english sources
]{biblatex}
\addbibresource{references.bib}  % the bib file to use
\DefineBibliographyStrings{german}{andothers = {{et\,al\adddot}}}  % replace u.a. with et al.
\useoutertheme{infolines2}
\colorlet{structure}{green!50!black}
\definecolor{tugreen}{RGB}{132,184,24}
\definecolor{tugrey}{RGB}{178,179,182}
\definecolor{tured}{RGB}{205,0,47}
\setbeamercolor{palette primary}{bg=tugreen,fg=black}
\setbeamercolor{palette secondary}{bg=tugrey!50!tugreen,fg=black}
\setbeamercolor{palette quaternary}{fg=black, bg=tugrey}
\setbeamercolor{caption name}{fg=tugreen}
\setbeamercolor{palette tertiary}{fg=black,bg=tugrey}


\setbeamercolor{palette compare}{bg=white!80!tugreen,fg=black}
\setbeamercolor{palette misc}{bg=white!80!tugreen,fg=black}
\setbeamercolor{palette white}{bg=white!99!black,fg=black}

\setbeamercolor{itemize item}{fg=tugreen}
\setbeamercolor{itemize subitem}{fg=tugreen}
\setbeamercolor{itemize subsubitem}{fg=tugreen}
\setbeamercolor{enumerate item}{fg=tugreen}
\setbeamercolor{enumerate subitem}{fg=tugreen}
\setbeamertemplate{itemize item}[square]
\setbeamercolor{title}{fg=black, bg=tugreen}
\setbeamercolor{frametitle}{fg=black, bg=tugreen}
\setbeamertemplate{navigation symbols}{}
\setbeamertemplate{frametitle}[default][center]


\setbeamertemplate{titlepage}{
\begin{center}
 \begin{beamercolorbox}[rounded=true, shadow=true, center,ht=0.75cm]{title}
 \begin{center}
 \usebeamerfont{title}\inserttitle
 \end{center}
 \end{beamercolorbox}
\vspace{1cm}
 \usebeamerfont{subtitle}\insertsubtitle \\
 \vspace{1cm}
 \usebeamerfont{author}\insertauthor \\
 \vspace{1cm}
 \insertinstitute
 \usebeamerfont{date}\insertdate
\begin{flushleft}\hspace*{4cm}%\includegraphics[width=5cm]{logos/tu.pdf}\end{flushleft}
\end{center}
}
\author{Jonah Blank}
\title{Indirect search for Dark Matter}
\date{28.11.2019}
\institute{TU Dortmund}
\begin{document}
%\begin{frame}[plain]
%\begin{tikzpicture}[remember picture,overlay]
%\node[at=(current page.center)] {
%%\includegraphics[width=1.2\paperwidth,height=\paperheight]{build/coma.pdf}
%};
%\end{tikzpicture}
%\end{frame}
\begin{frame}
\titlepage
\end{frame}
\begin{frame}[plain]
\begin{tikzpicture}[remember picture,overlay]
\node[at=(current page.center)] {
%\includegraphics[width=0.65\paperwidth,height=0.65\paperheight]{build/dir_ind_col.png}
};
\end{tikzpicture}
\footnotetext[1]{inspirehep.net}
\end{frame}
\begin{frame}
\tableofcontents
\end{frame}
\section{Hints for Dark Matter(DM)}
%\begin{frame}{Hints for Dark Matter(DM)}
%gravitational effect:
%\begin{itemize}
%\item existence of galaxy clusters
%\begin{itemize}
%\item Virial: $E_\text{kin}=-\left(\frac{1}{2}\right)\cdot E_\text{grav}$
%\item $E_\text{grav}$ way too small assuming only visible matter
%\end{itemize}\vfill
%\item galaxy rotation: velocity of outer rim
%\begin{itemize}
%\item distance $r$ from galactic center: $v_\text{rot}\propto\frac{1}{\sqrt{r}}$\vfill
%\item measured: $v_\text{rot}\approx$ const
%\end{itemize}\vfill
%\item fluctuations in cosmic microwave background (CMB)
%\begin{itemize}
%\item fluctuation of hot matter in early universe\vfill
%\item gravity $\leftrightarrow$ radiation pressure
%\end{itemize}
%\end{itemize}
%\end{frame}
\setbeamercolor{normal text}{fg=white,bg=black}
\renewcommand\footnoterule{}
\begin{frame}[plain]
\begin{tikzpicture}[remember picture,overlay]
\node[at=(current page.center)] {
%\includegraphics[width=1.2\paperwidth,height=\paperheight]{build/coma.pdf}
};
\end{tikzpicture}
\footnotetext[2]{arXiv:1604.00014}
\end{frame}
\renewcommand{\footnoterule}{%
  \kern -5pt
  \hrule width \textwidth 
  \kern 2pt
}
\setbeamercolor{normal text}{fg=black,bg=white}
\begin{frame}{Hints for Dark Matter(DM)}
gravitational effect:
\begin{itemize}
\item existence of galaxy clusters
\begin{itemize}
\item Virial: $\bar{E}_\text{kin}=-\left(\frac{1}{2}\right)\cdot \bar{E}_\text{grav}$\vfill
\item $E_\text{grav}$ way too small assuming only visible matter
\end{itemize}\vfill\pause
\item galaxy rotation: velocity of outer rim
\begin{itemize}
\item distance $r$ from galactic center: $v_\text{rot}\propto\frac{1}{\sqrt{r}}$\vfill
\item measured: $v_\text{rot}\approx$ const
\end{itemize}\vfill\pause
\item the cosmic microwave background (CMB)
\begin{itemize}
\item oscillation of hot matter in early universe\vfill
\item gravity $\leftrightarrow$ radiation pressure
\end{itemize}
\end{itemize}
\end{frame}
\begin{frame}{Hints for Dark Matter(DM)}
\begin{figure}
\begin{minipage}{0.45\textwidth}
fluctuations in CMB:
\begin{itemize}
\item oscillation of hot matter in early universe\vfill
\item gravity $\leftrightarrow$ radiation pressure\vfill
\item after cooling down red/blue shift due to gravitational field\\
$\rightarrow$ power spectrum
\end{itemize}
\end{minipage}
\begin{minipage}{0.53\textwidth}
%\includegraphics[keepaspectratio,width=0.9\textwidth]{build/cmb.pdf}
\end{minipage}
\end{figure}
\footnotetext[3]{Integrierter Kurs Physik 3, Prof. Tolan, Prof. Stolze, WS17/18, TU Dortmund}
\end{frame}
\begin{frame}{The CMB power spectrum - DM dependency}
\begin{figure}
\begin{subfigure}{0.45\textwidth}
%\includegraphics[width=0.68\textwidth]{build/clmatter/clmatter-001.png}
%\includegraphics[width=0.68\textwidth]{build/clmatter/clmatter-015.png}
\end{subfigure}
\begin{subfigure}{0.45\textwidth}\pause
%\includegraphics[width=0.75\textwidth]{build/clmatter/clmatter-004.png}\pause
\begin{itemize}
\item spectrum changes for different DM densities
\item peak ratio and decay rate to calculate amount of DM
\end{itemize}
\end{subfigure}\
\end{figure}
\footnotetext[4]{www.uchicago.edu}
\end{frame}
\begin{frame}{The CMB power spectrum - Planck measurement}
\begin{figure}
%\includegraphics[width=0.9\textwidth]{build/power.pdf}
\end{figure}
\footnotetext[5]{arXiv:1303.5075}
\end{frame}
\begin{frame}{The energy density distribution}
\begin{figure}
\begin{subfigure}{0.45\textwidth}
%\includegraphics[width=0.9\textwidth]{build/power.pdf}
\end{subfigure}
\begin{subfigure}{0.45\textwidth}
%\includegraphics[width=0.8\textwidth]{build/pie.pdf}
\end{subfigure}
\end{figure}\medskip
\begin{itemize}
\item only $5\%$ of energy in the universe explained by normal matter\medskip
\item 5x more DM than baryonic matter\medskip
\only<2>{\item But what \textbf{is} dark matter?}
\end{itemize}
\footnotetext[6]{arXiv:1303.5075}
\footnotetext[7]{Studies of dark matter annihilation and production in the Universe,\\ Carl Niblaeus (2019)}
\end{frame}
\setbeamercolor{normal text}{fg=white,bg=black}
\renewcommand\footnoterule{}
\begin{frame}[plain]
\begin{tikzpicture}[remember picture,overlay]
\node[at=(current page.center)] {
%\includegraphics[width=1\paperwidth,height=\paperheight]{build/types.png}
};
\end{tikzpicture}
\footnotetext[8]{cf. physics.aps.org}
\end{frame}
\renewcommand{\footnoterule}{%
  \kern -3pt
  \hrule width \textwidth
  \kern 2pt
}
\setbeamercolor{normal text}{fg=black,bg=white}
\section{Possible DM candidates}
\begin{frame}{Possible DM candidates}
\begin{itemize}
\item sterile neutrinos\footnotemark[9]
\begin{itemize}
\item additional right-handed neutrino eigenstate (ES), mass $\mathcal{O}(\si{\kilo\eV})$ \medskip
\item falling out of equilibrium at $T\geq m_{\nu} \rightarrow$ hot/warm thermal relic\medskip
\item mixing of mass ES (PMNS matrix)$\rightarrow$ small share of active ES\medskip
\item heavier (sterile) can decay into lighter (active) mass ES: $\nu_\text{s} \rightarrow \nu_\text{a} + \gamma$\medskip
\item photon emission line at $\si{\kilo\eV}$
\end{itemize}\vfill
\item super heavy gravitinos\footnotemark[10]
\begin{itemize}
\item 8 gravitinos, $m_\text{gr}\propto \SI{2e18}{\giga\eV} \rightarrow$ UHECR\medskip
\item participating in strong and EM interaction, $\alpha_\text{s,em}\propto \mathcal{O}(1)$\medskip
\item 2 color triplets - EM charge $\pm \frac{1}{3}$, 2 color singlets - EM charge $\pm \frac{2}{3}$\medskip
\item SM interaction only through annihilation
\end{itemize}
\end{itemize}
\footnotetext[9]{arXiv:1901.00151}
\footnotetext[10]{arXiv:1906.07262}
\end{frame}
\begin{frame}{\textbf{W}eakly \textbf{I}nteracting \textbf{M}assive \textbf{P}articles}
\begin{figure}
\begin{minipage}{0.65 \textwidth}
\begin{itemize}
\item WIMPs
\begin{itemize}
\item weak scale masses $\rightarrow$ search in gamma and cosmic rays\medskip
\item only interacting via weak interaction\medskip
\item annihilation cross section $\langle \sigma_\text{A}v\rangle\approx \SI{3e-26}{\centi\meter^3\per\second}$\medskip
\item thermally produced until \glqq freeze out\grqq\\
at $T<m_\text{WIMP}\rightarrow$ cold thermal relic\medskip
\item non-relativistic DM forming structures in the universe\medskip
\item $\Omega_\text{WIMP} = \frac{\rho_\text{WIMP}}{\rho_\text{c}}\propto 0.2$
\end{itemize}
\end{itemize}
\end{minipage}
\begin{minipage}{0.34\textwidth}
%\includegraphics[width=0.9\textwidth]{build/WIMP.pdf}
\end{minipage}
\end{figure}
\footnotetext[11]{nasa.gov}
\end{frame}
\begin{frame}{The thermal equilibrium}
\begin{minipage}{\textwidth}
\centering
%\includegraphics[width=0.5\textwidth]{build/equi.png}
\end{minipage}
\begin{minipage}{\textwidth}
\begin{itemize}
\item thermal equilibrium at $T\approx m_\text{DM}$\medskip
\item moment of freeze out important for modern structure of the universe
\item dependent on annihilation cross section
\end{itemize}
\end{minipage}
\footnotetext[12]{arXiv:1812.02029}
\end{frame}
%\begin{frame}{Possible DM candidates}
%\begin{itemize}
%\item super heavy DM
%\begin{itemize}
%\item cosmologically stable, low interaction rate\medskip
%\item not produced in thermodynamic equilibrium\medskip
%\item $m_\text{heavy}\propto \SI{e12}{\giga\eV} \rightarrow$ searched for in Ultra-High-Energy Cosmic Rays (UHECR)
%\end{itemize}\vfill
%\end{frame}
\setbeamercolor{normal text}{fg=white,bg=black}
\renewcommand\footnoterule{}
\begin{frame}[plain]
\begin{tikzpicture}[remember picture,overlay]
\node[at=(current page.center)] {
%\includegraphics[width=1.2\paperwidth,height=\paperheight]{build/bullet.jpg}
};
\end{tikzpicture}
\footnotetext[13]{chandra.harvard.edu}
\end{frame}
\renewcommand{\footnoterule}{%
  \kern -3pt
  \hrule width \textwidth
  \kern 2pt
}
\setbeamercolor{normal text}{fg=black,bg=white}
\section{Indirect searches for DM}
\subsection{Cosmic sources}
\begin{frame}{Cosmic sources for indirect detection}
\begin{figure}
\only<1>{\begin{minipage}{\textwidth}
\begin{itemize}
\item weak scale mass DM\\
$\rightarrow$ final states at SM and gamma ray energies\medskip
\item equal amounts of anti-/matter with max energy $m_\text{DM}$ each\\
in DM annihilation\medskip
\item photons experience no deflection in the universe\\
$\rightarrow$ spatial information to distinguish source
\end{itemize}
\end{minipage}}\vfill
\begin{minipage}{\textwidth}
\begin{small}
\begin{align*}
\Phi_\text{x,ann}(\Delta\Omega)&=\frac{1}{2}\frac{\mathrm{d}N_\text{x}}{\mathrm{d}E}\frac{\langle\sigma v \rangle}{4\pi m^2_\text{DM}}\cdot J_\text{ann}(\Delta\Omega)\\
J_\text{ann}(\Delta\Omega)&=\int_{\Delta\Omega}\int_\text{los}\rho^2_\text{DM}\mathrm{d}\ell\mathrm{d}\Omega
\end{align*}
\end{small}
\end{minipage}
\begin{minipage}{\textwidth}
\centering
\only<2>{%\includegraphics[width=0.8\textwidth]{build/j_fac.png}}
\only<3>{%\includegraphics[width=0.8\textwidth]{build/j_fac2.png}}
\end{minipage}
\end{figure}
\only<2-3>{\footnotetext[12]{arXiv:1812.02029}}
\end{frame}
\begin{frame}{Gamma Ray Detection: The Draco Spheroidal Dwarf Galaxy}
\begin{figure}
\begin{minipage}{0.48\textwidth}
\begin{itemize}
\item distance: 260,000 ly\medskip
\item diameter: 3000 ly\medskip
\item bright mass: 22,000,000 $M_{\odot}$\medskip
\item $J_\text{ann}\propto\SI{e18.8}{\giga\eV^2\per\centi\meter^5}$
%$\rightarrow \Phi_\gamma\approx \SI{5e-12}{\centi\meter^{-2}\second^{-1}}\left(\frac{\langle\sigma v\rangle}{\SI{2.2e-26}{\centi\meter^3\per\second}}\right)\left(\frac{\int\frac{\mathrm{d}N_\gamma}{\mathrm{d}E_\gamma}\mathrm{d}E_\gamma}{10}\right)\left(\frac{\SI{100}{\giga\eV}}{m_\text{DM}}\right)^2\left(\frac{J}{\SI{e18.8}{\giga\eV^2\per\centi\meter^5}}\right)$
\end{itemize}
\end{minipage}
\begin{minipage}{0.5\textwidth}
\centering
%\includegraphics[width=0.75\textwidth]{build/m87_ell.png}
\end{minipage}
\begin{minipage}{\textwidth}
\[
\rightarrow\Phi_\gamma\propto \SI{5e-12}{\centi\meter^{-2}\second^{-1}}\left(\frac{\SI{100}{\giga\eV}}{m_\text{DM}}\right)^2
\]
\uncover<2>{\begin{itemize}
\item Fermi Large Area Telescope effective area $\SI{0.85}{\meter^2}$\\
$\rightarrow$ 0.3 photons per year from DM annihilation detected
\end{itemize}}
\end{minipage}
\end{figure}
\footnotetext[14]{cornell.edu}
\end{frame}
\subsection{Current experiments}
\begin{frame}{Ground based vs. orbiting telescopes}
\begin{columns}[T]
\begin{column}{0.49\textwidth}
\begin{itemize}
\item ground based telescopes
\begin{itemize}
\item capture only part of the sky\medskip
\item have to take in account atmosphere\\
$\rightarrow$ only useful for high energy measurements\medskip
\item bigger detectors for better resolution
\end{itemize}
\end{itemize}
\hspace{1cm}%\includegraphics[width=0.9\textwidth]{build/CTA.png}
\end{column}
\begin{column}{0.49\textwidth}
\begin{itemize}
\item orbiting telescopes
\begin{itemize}
\item difficult maintenance\medskip
\item limited resolution and size\medskip
\item variable target of observation\medskip
\item cover lower energy spectra
\end{itemize}
\end{itemize}
\vspace{0.3cm}
\hspace{0.1cm}%\includegraphics[width=0.9\textwidth]{build/Fermi_LAT.jpg}
\end{column}
\end{columns}
\footnotetext[15]{cta-observatory.org}
\footnotetext[16]{nasa.gov}
\end{frame}
\begin{frame}{Current experiments}
\begin{figure}
\begin{minipage}{0.45\textwidth}
\only<1>{%\includegraphics[width=0.85\textwidth]{build/exp.pdf}}
\only<2-3>{%\includegraphics[width=0.85\textwidth]{build/exp2.pdf}}
\end{minipage}\pause
\begin{minipage}{0.52\textwidth}
\begin{itemize}
\item Alpha Magnetic Spectrometer: long term experiment at ISS\medskip \pause
\item AMS-02 collaboration presented new results in 2019
\end{itemize}
\end{minipage}\vfill
\begin{minipage}{\textwidth}
\begin{itemize}
\item confirming excess in high energetic positrons from previous measurements\medskip
\item new: similar distribution of anti-protons
\end{itemize}
\end{minipage}
\end{figure}
\footnotetext[17]{arXiv:1604.00014}
\end{frame}
\begin{frame}{\textbf{A}lpha \textbf{M}agnetic \textbf{S}pectrometer}
\begin{figure}
\begin{minipage}{0.8\textwidth}
%\includegraphics[keepaspectratio,width=0.9\textwidth]{build/ams_ges.pdf}
\end{minipage}
\begin{minipage}{\textwidth}
\begin{itemize}
\item internal data rate $\approx 7 \text{GB/s}$, transmission rate $\approx 2 \text{MB/s}$
\item 140 billion registered events since 2011
\end{itemize}
\end{minipage}
\end{figure}
\footnotetext[18]{inspirehep.net}
\end{frame}
\begin{frame}{\textbf{A}lpha \textbf{M}agnetic \textbf{S}pectrometer}
\begin{figure}
\begin{minipage}{0.5\textwidth}
%\includegraphics[width=0.8\textwidth]{build/ams_sch.pdf}
\end{minipage}
\begin{minipage}{0.48\textwidth}
\begin{itemize}
\item \textbf{T}ransition \textbf{R}adiation \textbf{D}etector: proton rejection\medskip
\item magnet \& 9 layers silicon tracker: particle momentum\medskip
\item \textbf{R}ing \textbf{I}maging \textbf{CH}erenkov: ion identification \& velocity measurement\medskip
\item \textbf{T}ime \textbf{O}f \textbf{F}light: particle mass\medskip
\item \textbf{E}lectromagnetic \textbf{Cal}orimeter: particle energy
\end{itemize}
\end{minipage}
\end{figure}
\footnotetext[19]{inspirehep.net}
\end{frame}
\setbeamercolor{normal text}{fg=white,bg=black}
\renewcommand\footnoterule{}
\begin{frame}[plain]
\begin{tikzpicture}[remember picture,overlay]
\node[at=(current page.center)] {
%\includegraphics[keepaspectratio,width=\textwidth]{build/ams_table.pdf}
};
\end{tikzpicture}
\footnotetext[20]{EPS-HEP Conference 2019}
\end{frame}
\renewcommand{\footnoterule}{%
  \kern -3pt
  \hrule width \textwidth
  \kern 2pt
}
\setbeamercolor{normal text}{fg=black,bg=white}
\begin{frame}{Positron excess in cosmic ray events}
\begin{minipage}{\textwidth}
%\includegraphics[keepaspectratio, width=0.75\textwidth]{build/positron.pdf}
\end{minipage}\vfill
\begin{minipage}{\textwidth}
\begin{itemize}
\item cosmic ray collisions only known origin of positrons
\end{itemize}
\end{minipage}
\footnotetext[21]{DOI:10.1103/PhysRevLett.122.041102}
\end{frame}
\begin{frame}{Positron excess in cosmic ray events}
\begin{minipage}{\textwidth}
%\includegraphics[keepaspectratio,width=0.75\textwidth]{build/positron_unk.png}
\end{minipage}\vfill
\begin{minipage}{\textwidth}
\begin{itemize}
\item maximum at $\mathcal{O}(\si{\giga\eV})$, flat tail towards higher energies\\
$\rightarrow$ excess in $0.1$-$\SI{1}{\tera\eV}$ positrons with cutoff $E_\text{s}$
\end{itemize}
\end{minipage}
\footnotetext[20]{cf. EPS-HEP Conference 2019}
\end{frame}
\begin{frame}{Positron excess in cosmic ray events}
\begin{minipage}{\textwidth}
%\includegraphics[keepaspectratio,width=0.75\textwidth]{build/positron_src.pdf}
\end{minipage}\vfill
\begin{minipage}{\textwidth}
\begin{itemize}
\item maximum at $\mathcal{O}(\si{\giga\eV})$, flat tail towards higher energies\\
$\rightarrow$ excess in $0.1$-$\SI{1}{\tera\eV}$ positrons with cutoff $E_\text{s}$
\end{itemize}
\end{minipage}
\footnotetext[21]{DOI:10.1103/PhysRevLett.122.041102}
\end{frame}
\begin{frame}{Positron excess in cosmic ray events}
\begin{minipage}{0.5\textwidth}
%\includegraphics[keepaspectratio, width=0.9\textwidth]{build/positron_src.pdf}
\end{minipage}
\begin{minipage}{0.4\textwidth}
\begin{tiny}
\[
\Phi_{\mathrm{e}^{+}}(E)=\frac{E^2}{\hat{E}^2}\left[C_\text{d}\left(\frac{\hat{E}}{E_\text{1}}\right)^{\gamma_\text{d}}+C_\text{s}\left(\frac{\hat{E}}{E_\text{2}}\right)^{\gamma_\text{s}}\mathrm{e}^{-\frac{\hat{E}}{E_\text{s}}}\right]
\]
\end{tiny}
\begin{itemize}
\item excess can be described adding a source term\medskip
\item established at $99,99\%$ CL with $E_\text{s}= 810^{+310}_{-180} \si{\giga\eV}$
\end{itemize}
\end{minipage}\vfill\pause
\begin{minipage}{\textwidth}
\begin{itemize}
\item pulsars can produce high energy positrons, but without sharp $E_\text{S}$\medskip
\item they do not produce anti-protons
\end{itemize}
\end{minipage}
\end{frame}
\begin{frame}{The anti-proton excess}
\begin{minipage}{\textwidth}
\centering
%\includegraphics[keepaspectratio,width=0.75\textwidth]{build/anti_proton.png}
\end{minipage}
\begin{minipage}{\textwidth}
\begin{itemize}
\item anti-protons follow similar distribution\medskip
\item could hint for DM annihilation:\\
equal production of matter/antimatter $\rightarrow$ sharp cut at $E\approx M_\text{DM}$
\end{itemize}
\end{minipage}
\footnotetext[20]{cf. EPS-HEP Conference 2019}
\end{frame}
\begin{frame}{Comparison of previous experiments}
\begin{minipage}{\textwidth}
\centering
%\includegraphics[keepaspectratio,width=0.75\textwidth]{build/results.png}
\end{minipage}\vfill
\begin{minipage}{\textwidth}
\begin{itemize}
\item confirming trends from LAT(2009), AMS-02(2013), PAMELA(2016) \medskip
\item first measurement of cutoff in $\si{TeV}$-positrons
\end{itemize}
\end{minipage}
\footnotetext[21]{DOI:10.1103/PhysRevLett.122.041102}
\end{frame}
\section{Conclusion and outlook}
\begin{frame}{Conclusion and outlook}
\begin{minipage}{0.42\textwidth}
conclusion:
\begin{itemize}
\item steady stream of new theories and scenarios\medskip
\item for now no unambiguous observations\medskip
\item background is mostly unknown $\rightarrow$ other explanations possible 
\end{itemize}
\end{minipage}
\begin{minipage}{0.57\textwidth}
%\includegraphics[keepaspectratio,width=0.75\textwidth]{build/expect.png}
\end{minipage}\vfill
\begin{minipage}{\textwidth}
outlook:
\begin{itemize}
\item extension of data taking period for AMS-02 $\rightarrow$ reduction of uncertainties\medskip
\item new upcoming experiments with better resolution: GAMMA-400, CTA
\end{itemize}
\end{minipage}
\footnotetext[20]{EPS-HEP Conference 2019}
\end{frame}
\end{document}

%♣☺∟♦☻♦☻☺®☺♠☺♥♦♠
